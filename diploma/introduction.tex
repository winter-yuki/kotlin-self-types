\section*{Введение}

Чтобы выдерживать серьёзную конкуренцию на рынке программного обеспечения (ПО), необходимо разрабатывать всё более сложные, расширяемые и поддерживаемые программы с минимальным количеством ошибок.
Поэтому разработка не обходится без использования сторонних библиотек и фреймворков, тщательного продумывания архитектуры и используемых абстракций.
Это всё инструменты, вбирающие в себя часть сложности, и делающие программу доступной для человеческого понимания, а значит, развития.

Однако не стоит забывать, что также немаловажную роль в успешности программного обеспечения играет выбор языка программирования (ЯП).
Он способен существенно упростить разработку за счёт предоставления достаточного количества разумно организованных возможностей.
Так, если ЯП реализует в себе популярные шаблоны программирования, у программиста не будет необходимости писать их вручную, рискуя допустить ошибку и чрезмерно усложнить программный интерфейс.
Также ЯП может предоставлять удобные средства для построения абстракций и контроля за корректностью кода, например, за счёт системы типов.
На успешности ПО сказывается и качество опыта программирования на проекте.
Если ЯП позволяет удобно и лаконично выражать задумку программиста, не требует написания большого количества boilerplate code, а так же сообщает о большом количестве ошибок ещё до этапа тестирования, то повышается эффективность разработки, и становится проще искать в команду квалифицированных специалистов.

В данной работе будет рассмотрено добавление в промышленный язык Kotlin Self-типов, позволяющих облегчить реализацию некоторых популярных шаблонов программирования путём уменьшения количества необходимого кода, и упрощения программных интерфейсов.
Для этого необходимо проанализировать подходы, используемые в других ЯП для достижения похожих целей.
Интегрировать Self-типы в типовую систему языка, убедиться, что она остаётся безопасной.
И наконец, реализовать новую функциональность в компиляторе.

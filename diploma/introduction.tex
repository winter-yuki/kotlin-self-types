\section*{Введение}

Чтобы выдерживать серьёзную конкуренцию на рынке программного обеспечения (ПО), необходимо разрабатывать всё более сложные, расширяемые и поддерживаемые программы с минимальным количеством ошибок.
Поэтому разработка не обходится без использования сторонних библиотек и фреймворков, тщательного продумывания архитектуры и используемых абстракций.
Эти все инструменты вбирают в себя часть сложности и делают программу доступной для человеческого понимания, а значит, развития.

Однако не стоит забывать, что также немаловажную роль в успешности программного обеспечения играет выбор языка программирования (ЯП).
Он способен существенно упростить разработку за счёт предоставления достаточного количества разумно организованных возможностей.
Так, если ЯП реализует в себе популярные шаблоны программирования, у программиста не будет необходимости писать их вручную, рискуя допустить ошибку и чрезмерно усложнить программный интерфейс.
Также ЯП может предоставлять удобные средства для построения абстракций и контроля за корректностью кода, например, за счёт системы типов.
%На разработке ПО сказывается и качество опыта программирования на проекте.
%Если ЯП позволяет удобно и лаконично выражать задумку программиста, не требует написания большого количества шаблонного кода и сообщает о существенной доле ошибок ещё до этапа тестирования, разработка становится проще, эффективнее и приносит больше положительных эмоций.
%В таком случае проще искать в команду увлечённых и квалифицированных специалистов.

В данной работе рассматривается добавление Self-типов в промышленный язык Kotlin, позволяющих облегчить реализацию некоторых популярных шаблонов программирования путём уменьшения количества необходимого кода, дополнительного типового контроля и упрощения программных интерфейсов.
Для этого, во-первых, в качестве мотивации были приведены многочисленные примеры прикладных сценариев использования Self-типов, в которых другие языковые решения оказываются менее оптимальными.
Во-вторых, были проанализированы теоретические результаты, из которых становятся понятны случаи нарушения безопасности системы с Self-типами.
В третьих, были рассмотрены существующие академические и практические решения по обеспечению безопасности системы с Self-типами.
В четвёртых, Self-типы были безопасным образом интегрированы в систему типов языка Kotlin на основании опыта других решений.
И наконец, Self-типы были реализованы в компиляторе kotlinc, а реализация была протестирована.

\section{Интеграция Self-типов в типовую систему языка Kotlin} \label{sec:integration}

Ранее в главе~\ref{sec:theory} были рассмотрены соображения, указывающие на потенциальную небезопасность неаккуратной реализации Self-типов.
Далее в главе~\ref{sec:impls} были приведены различные существующие решения по безопасному внедрению Self-типов в языки.
И теперь мы готовы к тому, чтобы интегрировать Self-типы в Kotlin, учитывая его специфику, и с использованием безопасных практик.

Так, сначала мы рассмотрим различные аспекты интеграции Self-типов в Kotlin.
Ключевой задачей является сохранение безопасности системы типов (опр.~\ref{def:sound}) вместе с поддержкой наибольшего количества полезных приложений Self-типов, описанных ранее в~\ref{subsec:applications}.
Поэтому далее приведём соответствие между построенными правилами и верифицированным формализмом, чтобы убедиться в сохранении безопасности обновлённой системы типов.
И наконец, обсудим реализацию прототипа Self-типов в компиляторе kotlinc.


\subsection{Основные понятия}

Рассмотрим основные понятия, которые будем использовать в рассуждениях о Self-типах в Kotlin.

\begin{definition}
    \label{def:bound}
    Границей Self-типа назовём наиболее общий тип ресивера, на котором может быть вызван соответствующий метод.
\end{definition}

\begin{minted}{kotlin}
    interface Base {
        fun base(): #\framebox{Self}# = #\framebox{this}#
    }

    class Defived : Base {
        fun derived() { /* ... */ }
    }

    fun test(d: Derived) = d.base() /* : Derived */ .derived()
\end{minted}

Граница Self-типа совпадает с типом ресивера текущей декларации метода.
В примере выше для \mintinline[escapeinside=??]{kotlin}|?\framebox{Self}?| bound'ом является \mintinline{kotlin}|Base| (обозначение \mintinline{kotlin}|Self(Base)|).

\begin{definition}
    \label{def:materialization}
    \term{Материализация Self-типа} --- подмена Self-типа в сигнатуре метода на тип ресивера в скоупе типа этого ресивера.
\end{definition}

Так для примера выше \texttt{Self} материализуется в \texttt{Derived} в его скоупе: \\
$(base: Base.() -> \framebox{Derived}) \in scope(Derived)$.

Скоупы Self-типа и его границы совпадают % TODO


\subsection{Синтаксис Self-типов}

% TODO


\subsection{Правила Self-типов}

% TODO

\subsubsection{Подтипизация}

% TODO

\subsubsection{Ближайшие супертипы}

% TODO

\subsubsection{Определение ближайшего супертипа}

% TODO


\subsection{Создание новых объектов Self-типа}

% TODO


\subsection{Позиции Self-типа}

% TODO


\subsection{Материализация Self-типа}

% TODO


\subsection{Соответствие формальным решениям}

% TODO


\subsection{Реализация прототипа Self-типов в компиляторе kotlinc}

% TODO

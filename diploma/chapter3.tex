\section{Анализ существующих реализаций Self-типов} \label{sec:impls}

Self-типы привлекают много внимания со стороны исследователей объектно ориентированного программирования, так как возникающее несоответствие наследования и подтипизации является вызовом для любого формализма~\cite{cook1989inheritance}.
Также, как было показано в разделе~\ref{subsec:applications}, Self-типы имеют множество полезных приложений.
Поэтому не удивительно, что Self-типы уже присутствуют во многих промышленных языках программирования.

В этой главе мы рассмотрим различные существующие подходы к реализации Self-типов в объектно ориентированных языках, чтобы выбрать для Kotlin наиболее удачные и, вместе с тем, безопасные решения.
Нас будут в первую очередь интересовать возможные значения Self-типа, позиции, в которых тот или иной язык позволяет использовать Self-типы, а также на меры, предпринимаемые языком для сохранения безопасности системы типов (опр.~\ref{def:sound}).


\subsection{Академические подходы} \label{subsec:academic-approaches}

Основным вопросом, занимающим исследователей, является возможность безопасного использования Self-типов в контравариантных позициях, а именно~--- в \term{бинарных методах} (методы, принимающие один параметр Self-типа).
Для этого было придумано множество подходов, рассмотрим основные из них.

Как было показано, если Self-тип встречается в методах класса в контравариантных позициях, то наследник не является подтипом~\cite{cook1989inheritance}.
Но существует другое отношение, отличное от подтипизации, которое тем не менее сохранятся.
Это отношение называют, например, \term{matching}~\cite{bruce1993safe}.
Если тип $A$ находится в отношении match с $B$ ($A \trianglelefteq B$), то все сообщения, которые можно послать объекту типа $B$, можно послать и объекту типа $A$.
Даже разрабатывались объектно ориентированные языки, полностью основанные на этом отношении, не содержащие подтипизации~\cite{bruce1997subtyping}.
Однако matching~--- существенно менее интуитивно понятное отношение, чем подтипизация, что приводило к разработке других отношений~\cite{ryu2016thistype}.
% TODO

Другим подходом является введение отдельной вселенной \term{точных типов (exact types)}, статический тип которых совпадает с типом значения во время исполнения (в противоположность неточным типам из~\ref{subsec:existentials}).
Бинарные методы разрешается вызывать только на точных типах, что фактически запрещает динамическую диспетчеризацию бинарных методов~\cite{bruce1997increasing}.
Этот подход требует дополнения системы типов выводом точных типов.
Расширения множества допустимых программ можно добиться, вводя в язык вспомогательные возможности, такие как локальное уточнение~\cite{saito2009matching}, точные типовые параметры, именованные wildcard'ы, конструкция утверждения точности~\cite{ryu2016thistype}.

Также существует некоторое количество подходов по созданию новых объектов Self-типа.
Сложность в том, чтобы гарантировать на этапе компиляции, что тип создаваемого объекта всегда является подтипом типа времени исполнения ресивера.
Для этого вводят \term{ненаследуемые методы} (которые обязаны быть переопределены в каждом наследнике)~\cite{saito2009matching}, а так же \term{виртуальные конструкторы}~\cite{ryu2016thistype}.


\subsection{TypeScript} \label{subsec:typescript}

В языке TypeScript~\cite{bierman2014understanding} Self-типы присутствуют под названием <<this-типы>>\footnote{\url{https://www.typescriptlang.org/docs/handbook/2/classes.html\#this-types}}.
У таких типов единственных обитатель~--- \mintinline{typescript}|this|.

Разработчики TypeScript не ставят задачи поддержки безопасности системы типов языка и this-типы можно использовать в произвольной позиции.
Поэтому нетрудно написать код, например, с this-типом в контравариантной позиции, который проходит проверку типов, но при этом получающий без явных приведений типов значение \mintinline{typescript}|undefined|.

\begin{minted}{typescript}
    class Box {
      content: string = "";
      sameAs(other: this): boolean {
        return other.content === this.content;
      }
    }

    class DerivedBox extends Box {
      otherContent: string = "?";
      sameAs(other: this): boolean {
        if (other.otherContent === undefined) {
          console.log("Система типов TS небезопасна")
        }
        return other.otherContent === this.otherContent;
      }
    }

    const base = new Box();
    const derived = new DerivedBox();

    function test(x: Box): boolean {
      return x.sameAs(base)
    }

    test(derived) // Печатает: "Система типов TS небезопасна"
\end{minted}


\subsection{Python}

Python~\cite{sanner1999python} --- изначально динамически типизированный язык программирования, в который последовательно добавляются возможности статической типизации.
В том числе в версию языка 3.11 были введены Self-типы\footnote{\url{https://peps.python.org/pep-0673/}\label{foot:self-pep}}.

Также авторы дизайн-предложения Self-типов для Python приводят\footref{foot:self-pep} интересную статистику, согласно которой паттерн типового параметра с рекурсивным ограничением (отчасти заменяемый Self-типами, как мы увидели выше в разделе~\ref{par:recursive-generics}) встречается в количестве 40\% случаев от использования других популярных типов~--- \mintinline{Python}|dict| и \mintinline{Python}|Callable|.

Стратегия реализации Self-типов в Python заключается в превращении их обратно в \mintinline{Python}|TypeVar| с рекурсивным ограничением\footref{foot:self-pep}.

Дизайн Self-типов в Python сходен с дизайном в TypeScript.
Только значение \mintinline{python}|self| населяет Self-тип, его можно использовать в произвольной позиции без гарантий безопасности со стороны системы типов.


\subsection{Java Manifold}

Плагин Manifold к компилятору Java позволяет\footnote{\url{https://github.com/manifold-systems/manifold/tree/master/manifold-deps-parent/manifold-ext\#the-self-type-with-self}} проаннотировать тип аннотацией \mintinline{Java}|@Self|.

В простой исходящей позиции аннотация \mintinline{Java}|@Self| позволяет обойтись без abstract override методов (см.~\ref{par:abstract-override}) и дополнительных приведений типов:

\begin{minted}{java}
    class VehicleBuilder {
        /* ... */
        public @Self VehicleBuilder withWheels(int wheels) {
            _wheels = wheels;
            return this;
        }
    }

    class AirplaneBuilder extends VehicleBuilder {
        /* ... */
    }

     Airplane airplane = new AirplaneBuilder()
        .withWheels(2) // Возвращает AirplaneBuilder
        .withWings(1)
\end{minted}

Во входящей позиции тоже можно использовать аннотацию \mintinline{Java}|@Self|.
Это позволяет иметь некоторые дополнительные типовые ограничения:

\begin{minted}{java}
    class A {
        public boolean equals(@Self Object obj) {
            /* ... */
        }
    }

    A a = new A();
    a.equals("строчка вместо объекта типа A"); // Ошибка компиляции
\end{minted}

Однако эти ограничения легко обойти.
Так, следующий код уже не отвергается системой типов.
Поэтому классическая проверка \mintinline{java}|isinstance| в методе \texttt{equals} всё ещё необходима.

\begin{minted}{java}
    Object obj = a;
    obj.equals("строчка вместо объекта типа A");
\end{minted}

\mintinline{Java}|@Self| можно безопасно использовать и в исходящей ковариантной позиции, например, для реализации рекурсивной структуры данных.
Однако для входящей позиции всё ещё требуется проверка времени исполнения.

\begin{minted}{java}
    public class Node {
        private List<Node> children;

        public List<@Self Node> getChildren() {
            return children;
        }

        public void addChild(@Self Node child) {
            checkAssignable(this, child); // Необходима проверка
            children.add(child);
        }
    }

    public class MyNode extends Node {
        /* ... */
    }

    MyNode myNode = findMyNode();
    List<MyNode> = myNode.getChildren();
\end{minted}

Manifold предоставляет возможность писать функции-расширения в Java, и в связи с этим, \mintinline{Java}|@Self| может ссылаться на тип ресивера функции-расширения.

\begin{minted}{java}
    public static <K,V> @Self Map<K,V> add(
            @This Map<K,V> thiz, K key, V value) {
        thiz.put(key, value);
        return thiz;
    }

    HashMap<String, String> map = new HashMap<>()
        .add("nick", "grouper")
        .add("miles", "amberjack");
\end{minted}


\subsection{Swift} \label{subsec:swift}

Swift~\cite{goodwill2015swift} --- компилируемый промышленный язык программирования со строгой статической типизацией.
Имеет безопасную поддержку Self-типов\footnote{\url{https://docs.swift.org/swift-book/documentation/the-swift-programming-language/types/\#Self-Type}}.

В классах Self-тип можно использовать только в простой исходящей позиции, значением Self-типа может быть только \mintinline{swift}|self|.

Специальный полиморфизм поддерживается в Swift через механизм протоколов.
Протоколы аналогичны Java интерфейсам, но не позволяют делать реализации методов по умолчанию.
В протоколах Self-тип можно использовать в произвольной позиции.

При реализации протокола классом, все Self-типы в позициях, отличных от простой исходящей должны быть заменены на тип текущего класса:

\begin{minted}{swift}
    protocol P {
        func produce() -> Self
        func consume(_ x: Self)
    }

    class C: P {
        func produce() -> Self { return self }
        func consume(_ x: #\framebox{C}#) {}
    }
\end{minted}

Протоколы можно использовать как ограничения типовых параметров, тогда Self-тип подменяется на этот типовой параметр.
Аналогично для \\\mintinline{swift}|some Protocol| параметров, так как для каждого из них генерируется типовой параметр с соответствующим ограничением.

\begin{minted}{swift}
    func testConstrait<T: P>(_ x: T) {
        x.consume(/* value of type T is expected */)
    }
\end{minted}

Также протоколы могут быть границами экзистенциальных типов: \\\mintinline{swift}|any Protocol|.
В таком случае вызов метода с Self-типом в позиции, отличной от простой исходящей запрещён:

\begin{minted}{swift}
    func testAny(_ x: any P) {
        // error: member 'consume' cannot be used on value of type
        // 'any P'; consider using a generic constraint instead
        x.consume(/* ... */)
    }
\end{minted}

Self-типы в функциях-расширениях ссылаются на расширяемый тип.

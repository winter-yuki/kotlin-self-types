\section{Интеграция Self-типов в типовую систему языка Kotlin} \label{sec:integration}

В данной главе будет дано описание интеграции Self-типов в типовую систему языка Kotlin.
Ключевой задачей является сохранение безопасности системы типов (опр.~\ref{def:sound}) вместе с поддержкой наибольшего количества полезных применений Self-типов, описанных ранее в~\ref{subsec:applications}.

%Для этого мы рассмотрим, какие значения могут иметь Self-тип вместе с правилами подтипизации и в каких позициях Self-тип безопасно использовать.

TODO % TODO

%Далее будем пользоваться следующими вспомогательными определениями.
%
%\begin{minted}{kotlin}
%    fun interface In<in T> {
%        fun accept(x: T)
%    }
%
%    fun interface Out<out T> {
%        fun produce(): T
%    }
%
%    fun interface Inv<T> {
%        fun id(x: T): T
%    }
%\end{minted}



























\subsection{Значения Self-типа}

TODO % TODO

\subsubsection{Небезопасные присваивания}

TODO % TODO

\subsubsection{Создание новых объектов Self-типа}

TODO % TODO

\subsection{Позиции Self-типа}

TODO % TODO

\subsubsection{Простая исходящая позиция}

TODO % TODO

\subsubsection{Простая входная позиция}

TODO % TODO

\subsubsection{Параметрическая исходящая позиция}

TODO % TODO

\subsubsection{Параметрическая входная позиция}

TODO % TODO

\subsubsection{Параметр наследуемого класса}

TODO % TODO

\subsubsection{Функции-расширения}

TODO % TODO

\subsection{Вызовы функций с Self-типами}

TODO % TODO

\subsection{Взаимодействие с Java}

TODO % TODO

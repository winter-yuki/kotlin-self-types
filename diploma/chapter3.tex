\section{Анализ существующих реализаций Self-типов}

%\begin{minted}[escapeinside=??]{kotlin}
%interface Base {
%    fun base(): ?\framebox{Self}? = ?\framebox{this}?
%}
%
%class Defived : Base {
%    fun derived() { /* ... */ }
%}
%
%fun test(d: Derived) = d.base() /* : ?\framebox{Derived}? */ .derived()
%\end{minted}

%\begin{definition}
%    \label{def:materialization}
%    \term{Материализация Self-типа} --- подмена Self-типа в сигнатуре метода на тип ресивера в скоупе типа этого ресивера.
%\end{definition}
%
%Так для примера выше \texttt{Self} материализуется в \texttt{Derived} в его скоупе: \\
%$(base: Base.() -> ?\framebox{Derived}?) \in scope(Derived)$
%
%\begin{definition}
%    \label{def:bound}
%    \term{Bound Self-типа} --- наиболее общий тип, в который Self-тип может быть материализован.
%\end{definition}
%
%Bound Self-типа совпадает с типом ресивера текущей декларации.
%В примере выше для \mintinline[escapeinside=??]{kotlin}|?\framebox{Self}?| bound'ом является \mintinline{kotlin}|Base| (обозначение \underline{\mintinline{kotlin}|Self(Base)|}).
%
%В данной главе будет дано описание интеграции Self-типов в типовую систему языка Kotlin.
%Ключевой задачей является сохранение безопасности системы типов (опр.~\ref{def:sound}) вместе с поддержкой наибольшего количества полезных применений Self-типов, описанных ранее в~\ref{subsec:applications}.
%
%%Для этого мы рассмотрим, какие значения могут иметь Self-тип вместе с правилами подтипизации и в каких позициях Self-тип безопасно использовать.
%
%TODO % TODO
%
%%Далее будем пользоваться следующими вспомогательными определениями.
%%
%%\begin{minted}{kotlin}
%%    fun interface In<in T> {
%%        fun accept(x: T)
%%    }
%%
%%    fun interface Out<out T> {
%%        fun produce(): T
%%    }
%%
%%    fun interface Inv<T> {
%%        fun id(x: T): T
%%    }
%%\end{minted}
%
%
%
%
%
%
%
%
%
%
%
%
%
%
%
%
%
%
%
%
%
%
%
%
%
%
%
%\subsection{Значения Self-типа}
%
%TODO % TODO
%
%\subsubsection{Небезопасные присваивания}
%
%TODO % TODO
%
%\subsubsection{Создание новых объектов Self-типа}
%
%TODO % TODO
%
%\subsection{Позиции Self-типа}
%
%TODO % TODO
%
%\subsubsection{Простая исходящая позиция}
%
%TODO % TODO
%
%\subsubsection{Простая входная позиция}
%
%TODO % TODO
%
%\subsubsection{Параметрическая исходящая позиция}
%
%TODO % TODO
%
%\subsubsection{Параметрическая входная позиция}
%
%TODO % TODO
%
%\subsubsection{Параметр наследуемого класса}
%
%TODO % TODO
%
%\subsubsection{Функции-расширения}
%
%TODO % TODO
%
%\subsection{Вызовы функций с Self-типами}
%
%TODO % TODO
%
%\subsection{Взаимодействие с Java}
%
%TODO % TODO

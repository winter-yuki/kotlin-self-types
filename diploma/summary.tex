\section*{Заключение}

В рамках данной работы был разработан дизайн Self-типов для языка Kotlin на основе опыта других языков, а так же была реализована поддержка Self-типов в компиляторе kotlinc.
Для этого были решены следующие задачи:
\begin{enumerate}
    \item Выделены особенности существующих реализаций Self-типов в других языках:
    \begin{enumerate}
        \item Какие значения можно типизировать Self-типом;
        \item В каких позициях разрешено использовать Self-тип;
        \item Каковы меры по обеспечению безопасности системы типов.
    \end{enumerate}
    \item Self-типы интегрированы\footnote{\url{https://github.com/winter-yuki/kotlin-self-types}} в типовую систему языка Kotlin:
    \begin{enumerate}
        \item Введено понятие материализации Self-типа;
        \item Прописаны правила подтипизации, вычисления супертипов;
        \item Разработаны условия типизации новых объектов Self-типом;
        \item Указаны безопасные позиции использования Self-типа.
    \end{enumerate}
    \item Поддержка Self-типов реализована\footnote{\url{https://github.com/winter-yuki/kotlin/tree/self-types}} в компиляторе kotlinc:
    \begin{enumerate}
        \item Добавлен новый вид типов --- Self-типы;
        \item Скоупы расширены функциональностью материализации Self-типов;
        \item Полученная реализация протестирована на предмет покрытия необходимых сценариев использования, а так же недопуска небезопасного кода с Self-типами.
    \end{enumerate}
\end{enumerate}

Приведённый дизайн Self-типов естественным образом дополняет систему типов языка Kotlin, расширяя её выразительные возможности на целый ряд полезных приложений.
А предложенная реализация органично встраивается в архитектуру компилятора kotlinc.
Таким образом, можно ожидать появления Self-типов в последующих версиях языка Kotlin.
